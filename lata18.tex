\documentclass{beamer}

%\usepackage{listings}
%\usepackage[francais]{babel}
\usepackage[T1]{fontenc}
\usepackage[utf8]{inputenc}
%\usepackage{MyriadPro}
\usepackage{cabin}
\usepackage{graphicx}
\usepackage{array}
\usepackage{tikz}
\usetikzlibrary{positioning, backgrounds, shapes, chains, decorations.pathmorphing}
\usepackage{forest}
\usepackage{amsmath,amsthm,amssymb}  
\usepackage{stmaryrd}
%\usepackage{mdsymbol}
\usepackage{MnSymbol}
\usepackage{xcolor}
\usepackage{verbatim}
\usepackage{array}
%\usepackage{csquotes}


\useoutertheme{infolines}

\input{macros}
\renewcommand{\hidden}[1]{}

%colors
\definecolor{darkgreen}{rgb}{0,0.5,0}
\usebeamercolor{block title}
\definecolor{beamerblue}{named}{fg}
\usebeamercolor{alert block title}
\definecolor{beamealert}{named}{fg}

\renewcommand{\colon}{\!:\!}


\newcommand\paraitem{%
 \quad
 \makebox[\labelwidth][r]{%
 \makelabel{%
 \usebeamertemplate{itemize \beameritemnestingprefix item}}}\hskip\labelsep}

\newcommand{\mmid}{\mathbin{{\mid}{\mid}}}

\begin{document}

\title{Efficient Translation with Linear Bimorphisms} 
\author[Antoine Venant]{Christoph Teichmann \and \underline{Antoine Venant} \and Alexander Koller}
\institute{Saarland University}
\date{\today}
\maketitle


\section{Introduction}

\begin{frame}
  \frametitle{General grammatical frameworks}
  \begin{itemize}
  \item Bringing different grammar formalisms within a common framework makes analyses and comparison easier.
  \item Makes common structure and differences explicit.
  \item Enable stating generic facts about grammar formalisms.  
  \item Particularly salient in computational linguistics (C.L.), given the variety of (synchronous) grammars and transducers.
  \item One desirable theoretical benefit: generic parsing (translation) complexity bounds.
  \item Use generic parsing/translation algorithm accross formalisms.
  \end{itemize}
\end{frame}

\begin{frame}
  \frametitle{The program}
  \begin{itemize}
  \item Brief overview of Interpreted Regular Tree Grammar (IRTG) [Koller \& Kuhlmann, '11], which offers such a general framework.
  \item Review the generic parsing algorithm and identify factors preventing a generic complexity analyses.
  \item Highlight the complexity problem.
  \item Solve the problem by reducing the grammar to specific kind of normal form, for which the complexity analysis is straightforward.
  \end{itemize}
\end{frame}

%\begin{frame}
%  \frametitle{What this talk is about -- Bimorphisms and IRTG}
%  \begin{itemize}
%  \item \textbf{Bimorphisms} offer a succint way to represent relation between trees.
%  %\item Model syntax-based relations between objects.
%  \item Two trees are related when both are homomorphic images of a common third one under two fixed distinct homomorphisms.
%  %\item Admissible derivation trees are compactly specified as a regular tree language.
%  \item Interpreting tree nodes as algebraic operations, one can represent relations between objects of arbitrary nature.
%  \item This yields \textbf{IRTG}s -- Interpreted Regular Tree Grammars.
%  \item IRTGs offer a common theoretical foundation to a wide range of grammatic formalisms relevant to Computational Linguistics.
%  \end{itemize}
%\end{frame}

\section{Bimorphisms, IRTG and translation.}

\begin{frame}
  \frametitle{An example regular tree grammar and derivation.}
  \begin{center}
    \begin{minipage}{0.49\linewidth}
    \[
    \overbrace{
      \begin{aligned}
        &E \rightarrow r_{a}(E,E) \\& E \rightarrow r_{m}(E,E)\\
        &E \rightarrow 1 \mid \dots \mid n\\
    \end{aligned}}^{G}
    \]
    \end{minipage}
    \begin{minipage}{0.49\linewidth}
      \begin{forest}
        [$r_{m}$ [ $r_{a}$ [$1$] [$2$]]  [$3$] ]
      \end{forest}
    \end{minipage}
  \end{center}
\end{frame}

\begin{frame}
  \frametitle{An example bimorphism and IRTG.}
    \begin{center}
      \begin{tabular}{cc}
        %$h_1$ & $h_2$\\ 
        $
        \begin{aligned}
          & r_a(x_1, x_2) \mapsto^{h_1} \cdot(\texttt{v}, \cdot(\cdot(\cdot(x_1, {+}), x_2) , \texttt{v}))\\
          & r_m(x_1, x_2) \mapsto^{h_1} \cdot(\cdot(x_1, {\times}), x_2)\\
          & i \rightarrow i
        \end{aligned}
        $ 
        &
        $
        \begin{aligned}
          & r_a(x_1, x_2) \mapsto^{h_2} \texttt{add}(x_1, x_2)\\
          & r_m(x_1, x_2) \mapsto^{h_2} \texttt{mul}(x_1, x_2)\\
          & i \rightarrow i
        \end{aligned}
        $
        \\
      \end{tabular}\\\vspace{1pt}
      
      \fontsize{6}{7}      
      \begin{minipage}{0.1\linewidth}
        $\texttt{v} 1 + 2\texttt{v} \times 3$
      \end{minipage}
    $\xleftarrow{\llbracket \cdot \rrbracket_{\Sigma^{\ast}}}$
    \begin{minipage}{0.18\linewidth}
	      \centering
	      \begin{forest}
      	      [$\cdot$, blue [$\cdot$, blue, edge=blue [ $\cdot$, red, edge=blue [ \texttt{v}, red, edge=red ] [$\cdot$, red, edge=red [$\cdot$, red, edge=red [$\cdot$, red, edge=red [1, edge=red] [+, red, edge=red]] [2, edge=red] ] [\texttt{v}, red, edge=red] ]  ] [ $\times$, blue, edge=blue] ] [3, edge=blue] ]
           \end{forest}
      \end{minipage}
      $\xleftarrow{h_1}$      
	  \begin{minipage}{0.1\linewidth}
		  \centering
		  \begin{forest}
		    [$r_{m}$, blue [ $r_{a}$, red, edge=blue [$1$, edge=red] [$2$, edge=red] ] [$3$, edge=blue] ] 
		  \end{forest}
	  \end{minipage}
      $\xrightarrow{h_2}$
      \begin{minipage}{0.18\linewidth}
      	\centering
      	\begin{forest}
      		[\texttt{mul}, blue [ \texttt{add},red, edge=blue [$1$, edge=red] [$2$, edge=red] ]  [$3$, edge=blue] ]
      	\end{forest}
      \end{minipage}
      $\xrightarrow{\llbracket \cdot \rrbracket_{\mathbb{N}}}$
      \begin{minipage}{0.05\linewidth}
      	\centering
      	$9$
      \end{minipage}
    \end{center}
\end{frame}

\begin{frame}
  \frametitle{The program.}
  \begin{itemize}
  \item The image of a regular language through a bimorphism can be computed.
  \item \alert{But how efficiently can this be done, w.r.t. different shapes of homomorphisms?}
  \item Also forms the basis of a generic 2-step algorithm to translate an input object using an IRTG:
    \begin{enumerate}
    \item Find the set of algebraic decompositions of the input object.
    \item Translate this set of decompositions through the underlying bimorphism.
    \end{enumerate}
  \item \alert{step 1. is the algebra-dependent part. Can step 2 add to the general complexity of translating?}
  \item We answer these questions for the class of bimorphisms with \emph{linear} homomorphisms over \emph{binary} signatures ($LB$-bimorphisms).
  \end{itemize}
\end{frame}

\begin{frame}{Further motivation from C.L.}
  \begin{itemize}
  \item In Computational Linguistics, one commonly deals with very large FTAs (millions of rules).
  \item Knowing whether the shape of homomorphisms induce a complexity overhead is therefore a pressing issue.
  \item For a range of known formalisms encodable as IRTGs, translation complexity has been obversed to be exactly the complexity of algebraic decomposition.
  \item Does this generalize?
  \item If I design a new grammatical formalism using IRTGs and a known algebra, can I immediately conclude to a complexity upper bound?
  \end{itemize}
\end{frame}

\begin{frame}
  \frametitle{Translation problems.}

  \begin{block}{For a bimorphism $\langle G, h_1, h_2 \rangle$}
    Given as input a finite tree automaton (FTA) $F$, compute (a compact representation of)
    \[ \{ \langle h_1(t), h_2(t) \rangle \mid t \in L(G) \textnormal{ and } h_1(t) \in L(F) \} \]
  \end{block}
  \begin{itemize}
  \item Solve this by computing an automaton recognizing \[h_1^{-1}(L(F)) \cap L(G)\].
  \item \alert{How expensive is computing $h_1^{-1}(F)$ (Invhom problem)?}
  \end{itemize}
\end{frame}

\begin{frame}
  \frametitle{Translation problems (cont'd)}
  \begin{block}{For an IRTG $\langle G, h_1, h_2, \mathcal{A}_1 \triangleeq \langle D_1, \llbracket \cdot \rrbracket_1 \rangle , \mathcal{A}_2 \triangleeq \langle D_2, \llbracket \cdot \rrbracket_2 \rangle  \rangle$}
    Given as input an object $o \in D_1$, compute (a compact representation of)
    \[ \{ \langle \llbracket h_1(t) \rrbracket_1, \llbracket h_2(t) \rrbracket_2 \rangle \mid t \in L(G) \textnormal{ and } \llbracket h_1(t) \rrbracket_1 = o \} \]
  \end{block}
  \begin{itemize}
  \item Solve this in two steps:
    \begin{enumerate}
    \item (assuming feasible) compute a decomposition automaton $D_o$ recognizing $\{ t \mid \llbracket t \rrbracket_1 = o \}$
    \item Solve the translation problem for $\langle G, h_1, h_2 \rangle$ over $D_o$.
    \end{enumerate}
  \item  \alert{Can we express the complexity of this procedure in terms of $|D_o|$ only?}
  \item Parsing and recognazibility problems definable and reducible to translation.
  \end{itemize}  
\end{frame}

\section{Cost of the invhom.}

\begin{frame}
  \frametitle{Reminder}
    \begin{center}
      \begin{tabular}{cc}
        %$h_1$ & $h_2$\\ 
        $
        \begin{aligned}
          & r_a(x_1, x_2) \mapsto^{h_1} \cdot(\texttt{v}, \cdot(\cdot(\cdot(x_1, {+}), x_2) , \texttt{v}))\\
          & r_m(x_1, x_2) \mapsto^{h_1} \cdot(\cdot(x_1, {\times}), x_2)\\
          & i \rightarrow i
        \end{aligned}
        $ 
        &
        $
        \begin{aligned}
          & r_a(x_1, x_2) \mapsto^{h_2} \texttt{add}(x_1, x_2)\\
          & r_m(x_1, x_2) \mapsto^{h_2} \texttt{mul}(x_1, x_2)\\
          & i \rightarrow i
        \end{aligned}
        $
        \\
      \end{tabular}\\\vspace{1pt}
      
      \fontsize{6}{7}      
      \begin{minipage}{0.1\linewidth}
        $\texttt{v} 1 + 2\texttt{v} \times 3$
      \end{minipage}
    $\xleftarrow{\llbracket \cdot \rrbracket_{\Sigma^{\ast}}}$
    \begin{minipage}{0.18\linewidth}
	      \centering
	      \begin{forest}
      	      [$\cdot$, blue [$\cdot$, blue, edge=blue [ $\cdot$, red, edge=blue [ \texttt{v}, red, edge=red ] [$\cdot$, red, edge=red [$\cdot$, red, edge=red [$\cdot$, red, edge=red [1, edge=red] [+, red, edge=red]] [2, edge=red] ] [\texttt{v}, red, edge=red] ]  ] [ $\times$, blue, edge=blue] ] [3, edge=blue] ]
           \end{forest}
      \end{minipage}
      $\xleftarrow{h_1}$      
	  \begin{minipage}{0.1\linewidth}
		  \centering
		  \begin{forest}
		    [$r_{m}$, blue [ $r_{a}$, red, edge=blue [$1$, edge=red] [$2$, edge=red] ] [$3$, edge=blue] ] 
		  \end{forest}
	  \end{minipage}
      $\xrightarrow{h_2}$
      \begin{minipage}{0.18\linewidth}
      	\centering
      	\begin{forest}
      		[\texttt{mul}, blue [ \texttt{add},red, edge=blue [$1$, edge=red] [$2$, edge=red] ]  [$3$, edge=blue] ]
      	\end{forest}
      \end{minipage}
      $\xrightarrow{\llbracket \cdot \rrbracket_{\mathbb{N}}}$
      \begin{minipage}{0.05\linewidth}
      	\centering
      	$9$
      \end{minipage}
    \end{center}
\end{frame}

\begin{frame}
  \frametitle{The invhom problem.}

  \begin{itemize}
  \item Going back to the minimal 'arithmetic' bimorphism.
  \item Consider $F$ with axiom $S$ and $O(n)$ rules:
    \[
    \begin{aligned}
      & S \rightarrow \cdot(Y, N_i) \quad (i \in [1,n])\\
      & Y \rightarrow \cdot(N_i, M) \quad (i \in [1,n])\\
      & M \rightarrow \times\\
      & N_i \rightarrow i \quad (i \in [1,n])\\ 
    \end{aligned}    
    \]
    ($\llbracket L(F) \rrbracket_{\Sigma^\ast}$ is the set of strings of the form $i \times j$ for some $0 \le i,j \le n $).
  \end{itemize}
\end{frame}

\begin{frame}
  \frametitle{The invhom problem (cont'd)}
  \[
    \begin{aligned}
      & S \rightarrow \cdot(Y, N_i) \quad (i \in [1,n])\\
      & Y \rightarrow \cdot(N_i, M) \quad (i \in [1,n])\\
      & M \rightarrow \times\\
      & N_i \rightarrow i \quad (i \in [1,n])\\ 
    \end{aligned}    
    \]
    
    \begin{itemize}
    \item Apply generic algorithm to compute the image of $F$ (\emph{i.e.} the set $[1, n^2]$).
    \item Need to compute an automaton for $h_1^{-1}(L(F))$.
    \item Standard construction (top-down): for each label $l$ in $G$ and state $X$ in $F$, run $F$ from $X$ on $h_1(l)$. If this succeeds and reaches children states $Y_1, \dots, Y_n$, add a rule $X \rightarrow l(Y_1, \dots, Y_n)$ to the invhom automaton. 
    \end{itemize}
\end{frame}

\begin{frame}
  \frametitle{The invhom problem (cont'd)}
  \[
  \begin{aligned}
    & S \rightarrow \cdot(Y, N_i) \quad (i \in [1,n])\\
    & Y \rightarrow \cdot(N_i, M) \quad (i \in [1,n])\\
    & M \rightarrow \times\\
    & N_i \rightarrow i \quad (i \in [1,n])\\ 
  \end{aligned}    
  \]

  \begin{itemize}
  \item Recall $r_m(x_1, x_2) \mapsto^{h_1} \cdot(\cdot(x_1, {\times}), x_2)$.
  \item We have $S \rightarrow^{*}_{F} \cdot(\cdot(N_i, {\times}), N_j)$ for $i,j \in [1,n]$.
  \item Hence we add $S \rightarrow r_m(N_i, N_j)$ to the invhom for $i,j \in [1,n]$.
  \item \alert{$O(n^2)$ rules v.s. $O(n)$ rules for $F$!}.
  \item Still, $[1,n^2]$ can be described in the arithmetic signature with only $n$ rules, just as $F$.
  \end{itemize}
    
\end{frame}

\section{Normal Form}


\begin{frame}
  \frametitle{Reminder}
    \begin{center}
      \begin{tabular}{cc}
        %$h_1$ & $h_2$\\ 
        $
        \begin{aligned}
          & r_a(x_1, x_2) \mapsto^{h_1} \cdot(\texttt{v}, \cdot(\cdot(\cdot(x_1, {+}), x_2) , \texttt{v}))\\
          & r_m(x_1, x_2) \mapsto^{h_1} \cdot(\cdot(x_1, {\times}), x_2)\\
          & i \rightarrow i
        \end{aligned}
        $ 
        &
        $
        \begin{aligned}
          & r_a(x_1, x_2) \mapsto^{h_2} \texttt{add}(x_1, x_2)\\
          & r_m(x_1, x_2) \mapsto^{h_2} \texttt{mul}(x_1, x_2)\\
          & i \rightarrow i
        \end{aligned}
        $
        \\
      \end{tabular}\\\vspace{1pt}
      
      \fontsize{6}{7}      
      \begin{minipage}{0.1\linewidth}
        $\texttt{v} 1 + 2\texttt{v} \times 3$
      \end{minipage}
    $\xleftarrow{\llbracket \cdot \rrbracket_{\Sigma^{\ast}}}$
    \begin{minipage}{0.18\linewidth}
	      \centering
	      \begin{forest}
      	      [$\cdot$, blue [$\cdot$, blue, edge=blue [ $\cdot$, red, edge=blue [ \texttt{v}, red, edge=red ] [$\cdot$, red, edge=red [$\cdot$, red, edge=red [$\cdot$, red, edge=red [1, edge=red] [+, red, edge=red]] [2, edge=red] ] [\texttt{v}, red, edge=red] ]  ] [ $\times$, blue, edge=blue] ] [3, edge=blue] ]
           \end{forest}
      \end{minipage}
      $\xleftarrow{h_1}$      
	  \begin{minipage}{0.1\linewidth}
		  \centering
		  \begin{forest}
		    [$r_{m}$, blue [ $r_{a}$, red, edge=blue [$1$, edge=red] [$2$, edge=red] ] [$3$, edge=blue] ] 
		  \end{forest}
	  \end{minipage}
      $\xrightarrow{h_2}$
      \begin{minipage}{0.18\linewidth}
      	\centering
      	\begin{forest}
      		[\texttt{mul}, blue [ \texttt{add},red, edge=blue [$1$, edge=red] [$2$, edge=red] ]  [$3$, edge=blue] ]
      	\end{forest}
      \end{minipage}
      $\xrightarrow{\llbracket \cdot \rrbracket_{\mathbb{N}}}$
      \begin{minipage}{0.05\linewidth}
      	\centering
      	$9$
      \end{minipage}
    \end{center}
\end{frame}

\begin{frame}
  \frametitle{Sketch of the solution.}
  \begin{itemize}
  \item We formalize a solution for the class of $LB$-bimorphism -- bimorphisms as above involving linear homomorphisms, with symbol of max-arity $2$.
  \item Classically encountered in C.L.
  \end{itemize}

  
  We build on the following:
  \begin{enumerate}
  \item Solutions to the translation problem are invariant through weak-equivalence $\rightarrow$ one can transform the grammar.
  \item If all homomorphic images $h(l)$ have height at most $1$ , it is trivial that the standard construction of the invhom $h^{-1}(L(F))$ has at most as many rules as $F$.
  \end{enumerate}
\end{frame}

\begin{frame}
  \frametitle{Sketch of the solution (cont'd).}
  
  \begin{block}{Thm}
    for every $LB$-bimorphism (resp. IRTG), there exists an $LB$-bimorphism (resp. IRTG) with same relation and in which every homomorphic images have height at most 1. 
  \end{block}

  \begin{itemize}
  \item We call such an equivalent $LB$-bimorphism (IRTG) a truncated normal form (\textbf{TNF}).
  \item We provide an algorithm bringing an arbitrary $LB$-bimorphism (IRTG) into TNF and prove its correctness.
  \end{itemize}
 
  
\end{frame}

\section{Bringing $LB$-bimorphisms into TNF.}

\begin{frame}
  \frametitle{Sketch}

  \begin{itemize}
  \item The procedure consists in an elementary truncation step applied until saturation.
  \item Each truncation step targets a single homomorphism $h$ and a single rule $r$ with an homomorphic image of height $ > 1$.
  \item The truncation step `splits' $r$ and into several `fresh' labels, adding corresponding rules to the grammar.
  \item $h(r)$ is distributed over the new label thereby lessening the maximal images' height.
  \item One need to carefully adapt other homomorphisms, to prevent modifying the translation relation.
  \end{itemize}
\end{frame}

\begin{frame}
  \frametitle{Example -- 2 case}
  
  \begin{center}
    \begin{minipage}{0.49\linewidth}
      \[
        \begin{aligned}
          &E \rightarrow r_{a}(E,E) \quad \underbrace{E \rightarrow r_{m}(E,E)}_{{\color{blue} \cdot ({\color{red}\cdot(x_1, {\times})}, {\color{darkgreen}x_2})}}\\
          &E \rightarrow 1 \mid \dots \mid n\\
      \end{aligned}
      \]
    \end{minipage}
    \begin{minipage}{0.49\linewidth}
      \begin{forest}
        [$r_{m}$ [ $r_{a}$ [$1$] [$2$]]  [$3$] ]
      \end{forest}
    \end{minipage}
  \end{center}
  
    \begin{itemize}
    \item Target $r_{m}$ and $h_1$.
    \end{itemize}

    \begin{center}
    \begin{minipage}{0.49\linewidth}
      \[
        \begin{aligned}
          &E \rightarrow r_{a}(E,E) \\
          &\underbrace{E \rightarrow r^1_{m}(M^2,M^3)}_{\color{blue}\cdot(x_1, x_2)}\\
          &\underbrace{M^2 \rightarrow r_m^2(E)}_{{\color{red}\cdot(x_1, {\times})}} \quad \underbrace{M^3 \rightarrow r_m^3(E)}_{\color{darkgreen} x_1}\\
          &E \rightarrow 1 \mid \dots \mid n\\
      \end{aligned}
      \]
    \end{minipage}
    \begin{minipage}{0.49\linewidth}
      \begin{forest}
        [$r^1_{m}$ [ $r^2_m$ [ $r_{a}$ [$1$] [$2$]] ]  [ $r^3_m$ [$3$]] ]
      \end{forest}
    \end{minipage}
    \end{center}    
\end{frame}

\begin{frame}
  \frametitle{What about the unselected $h_2$?}
  
  \begin{center}
    \begin{minipage}{0.49\linewidth}
      \[
        \begin{aligned}
          &E \rightarrow r_{a}(E,E) \quad \overbrace{E \rightarrow r_{m}(E,E)}^{{\color{blue} \texttt{mult}(x_1, x_2)}}\\
          &E \rightarrow 1 \mid \dots \mid n\\
      \end{aligned}
      \]
    \end{minipage}
    \begin{minipage}{0.49\linewidth}
      \begin{forest}
        [$r_{m}$ [ $r_{a}$ [$1$] [$2$]]  [$3$] ]
      \end{forest}
    \end{minipage}
  \end{center}
  
    \begin{itemize}
    \item Move the whole image to $r_m^1$ which references both children.
    \end{itemize}

    \begin{center}
    \begin{minipage}{0.49\linewidth}
      \[
        \begin{aligned}
          &E \rightarrow r_{a}(E,E) \\
          &\overbrace{E \rightarrow r^1_{m}(M^2,M^3)}^{{\color{blue} \texttt{mult}(x_1, x_2)}}\\
          &\overbrace{M^2 \rightarrow r_m^2(E)}^{ x_1} \quad \overbrace{M^3 \rightarrow r_m^3(E)}^{x_1}\\
          &E \rightarrow 1 \mid \dots \mid n\\
      \end{aligned}
      \]
    \end{minipage}
    \begin{minipage}{0.49\linewidth}
      \begin{forest}
        [$r^1_{m}$ [ $r^2_m$ [ $r_{a}$ [$1$] [$2$]] ]  [ $r^3_m$ [$3$]] ]
      \end{forest}
    \end{minipage}
    \end{center}
\end{frame}

\begin{frame}
  \frametitle{Transformation so far:}
  \begin{center}
  \fontsize{6}{7}      
  \begin{minipage}{0.1\linewidth}
    $\texttt{v} 1 + 2\texttt{v} \times 3$
  \end{minipage}
  $\xleftarrow{\llbracket \cdot \rrbracket_{\Sigma^{\ast}}}$
  \begin{minipage}{0.18\linewidth}
    \centering
    \begin{forest}
      [$\cdot$, blue [$\cdot$, red, edge=blue [ $\cdot$, edge=red [ \texttt{v} ] [$\cdot$ [$\cdot$ [$\cdot$ [1] [+]] [2] ] [\texttt{v}] ]  ] [ $\times$, red, edge=red] ] [3, edge=blue] ]
    \end{forest}
  \end{minipage}
  $\xleftarrow{h_1}$      
  \begin{minipage}{0.1\linewidth}
    \centering
    \begin{forest}
      [$r^1_{m}$, blue [ $r^2_{m}$, red, edge=blue  [$r_{a}$, edge=red [$1$] [$2$]]] [$r^3_{m}$, darkgreen, edge=blue [$3$, edge=darkgreen]]] 
    \end{forest}
  \end{minipage}
  $\xrightarrow{h_2}$
  \begin{minipage}{0.18\linewidth}
      	\centering
      	\begin{forest}
      	  [\texttt{mul}, blue [ \texttt{add}, edge=blue [$1$] [$2$] ]  [$3$, edge=blue] ]
      	\end{forest}
  \end{minipage}
  $\xrightarrow{\llbracket \cdot \rrbracket_{\mathbb{N}}}$
  \begin{minipage}{0.05\linewidth}
    \centering
    $9$
  \end{minipage}
  \end{center}
\end{frame}

\begin{frame}
  \frametitle{Example -- 1 case}
  
    \begin{minipage}{0.49\linewidth}
      \[
      \begin{aligned}
        &\underbrace{E \rightarrow r_{a}(E,E)}_{{\color{blue} \cdot({\color{red} \texttt{v}}, {\color{darkgreen} \cdot(\cdot(\cdot(x_1, {+}), x_2) , \texttt{v})})}}\\
      \end{aligned}
      \]
    
    \begin{itemize}
    \item Target $r_{m}$ and $h_1$.
    \end{itemize}

   
      \[
      \begin{aligned}
        &\underbrace{E \rightarrow r^1_{a}(A^2, A^3)}_{\color{blue}\cdot(x_1, x_2)}\\
        &\underbrace{A^2 \rightarrow r_a^2}_{{\color{red}\texttt{v}}} \quad \underbrace{A^3 \rightarrow r_a^3(E, E)}_{{\color{darkgreen} \cdot(\cdot(\cdot(x_1, {+}), x_2) , \texttt{v})}}\\
      \end{aligned}
      \]
    \end{minipage}
    \begin{minipage}{0.49\linewidth}
      \begin{center}
        \begin{forest}
          [$r^1_{m}$ [ $r^2_m$ [$r^1_{a}$ [$r^2_a$] [$r^3_a$ [$1$] [$2$]] ]] [$r^3_m$ [$3$]] ]
        \end{forest}
      \end{center}
  \end{minipage}
\end{frame}


\begin{frame}
  \frametitle{Unselected homomorphism $h_2$}
  
    \begin{minipage}{0.49\linewidth}
      \[
      \begin{aligned}
        &\overbrace{\underbrace{E \rightarrow r_{a}(E,E)}_{{\color{blue} \cdot({\color{red} \texttt{v}}, {\color{darkgreen} \cdot(\cdot(\cdot(x_1, {+}), x_2) , \texttt{v})})}}}^{{\color{darkgreen} \texttt{add}(x_1, x_2) }}\\
      \end{aligned}
      \]
    
    \begin{itemize}
    \item Again, affect the whole image to rule referencing both children, here, $r_a^3$.
    \end{itemize}

   
      \[
      \begin{aligned}
        &\overbrace{\underbrace{E \rightarrow r^1_{a}(A^2, A^3)}_{\color{blue}\cdot(x_1, x_2)}}^{x_2}\\
        &\overbrace{\underbrace{A^2 \rightarrow r_a^2}_{{\color{red}\texttt{v}}}}^{\bot} \quad \overbrace{\underbrace{A^3 \rightarrow r_a^3(E, E)}_{{\color{darkgreen} \cdot(\cdot(\cdot(x_1, {+}), x_2) , \texttt{v})}}}^{{\color{darkgreen} \texttt{add}(x_1, x_2)}}\\
      \end{aligned}
      \]
    \end{minipage}
    \begin{minipage}{0.49\linewidth}
      \begin{center}
        \begin{forest}
          [$r^1_{m}$ [ $r^2_m$ [$r^1_{a}$ [$r^2_a$] [$r^3_a$ [$1$] [$2$]] ]] [$r^3_m$ [$3$]] ]
        \end{forest}
      \end{center}
  \end{minipage}
\end{frame}

\begin{frame}
  \frametitle{Transformation so far:}
  \begin{center}
  \fontsize{6}{7}      
  \begin{minipage}{0.1\linewidth}
    $\texttt{v} 1 + 2\texttt{v} \times 3$
  \end{minipage}
  $\xleftarrow{\llbracket \cdot \rrbracket_{\Sigma^{\ast}}}$
  \begin{minipage}{0.18\linewidth}
    \centering
    \begin{forest}
      [$\cdot$, [$\cdot$ [ $\cdot$, blue [ \texttt{v}, red, edge=blue ] [$\cdot$, darkgreen, edge=blue [$\cdot$, darkgreen, edge=darkgreen [$\cdot$, darkgreen, edge=darkgreen [1, edge=darkgreen] [+, darkgreen, edge=darkgreen]] [2, edge=darkgreen] ] [\texttt{v}, darkgreen, edge=darkgreen] ]  ] [ $\times$] ] [3] ]
    \end{forest}
  \end{minipage}
  $\xleftarrow{h_1}$      
  \begin{minipage}{0.1\linewidth}
    \centering
    \begin{forest}
       [$r^1_{m}$ [ $r^2_m$ [$r^1_{a}$, blue [$r^2_a$, red, edge=blue] [$r^3_a$, darkgreen, edge=blue [$1$, edge=darkgreen] [$2$, edge=darkgreen]] ]] [$r^3_m$ [$3$]] ]
    \end{forest}
  \end{minipage}
  $\xrightarrow{h_2}$
  \begin{minipage}{0.18\linewidth}
      	\centering
      	\begin{forest}
      	  [\texttt{mul} [ \texttt{add}, darkgreen [$1$, edge=darkgreen] [$2$, edge=darkgreen] ]  [$3$] ]
      	\end{forest}
  \end{minipage}
  $\xrightarrow{\llbracket \cdot \rrbracket_{\mathbb{N}}}$
  \begin{minipage}{0.05\linewidth}
    \centering
    $9$
  \end{minipage}
  \end{center}
\end{frame}

\begin{frame}
  \frametitle{The general picture -- 2 case}
  \centering
\begin{tabular}{|c|c|c|}
\hline
& Selected Homomorphism & Unselected Homomorphism \\
\hline
\parbox{0.1\textwidth}{Before} &
\begin{minipage}{0.33 \textwidth}
%\resizebox{1\columnwidth}{3cm}
{
\begin{tikzpicture}[scale=0.5, every node/.style={scale=0.5}]
	\tikzset{decoration={snake,amplitude=1mm,segment length=5mm,
	                       post length=1mm,pre length=1mm}}

	 \node (image) at (-3,-1) {$h_1(r) =$};

	 \node (root) at (0,0) {};
	 \node (left) at (-3,-3) {};
	 \node (right) at (3,-3) {};
	 
	 \node (top) at (0,-0.4) {f};
	 \node (internalL) at (-1,-1.75) {$t_1$};
	 \node (internalR) at (1,-1.75) {$t_2$};	  
	 
	 \node (lml) at (-2,-3) {};
	 \node (lmm) at (-1.5,-2.5) {};
	 \node (lmr) at (-1,-3) {};
	 
	 \node (rml) at (1,-3) {};
	 \node (rmm) at (1.5,-2.5) {};
	 \node (rmr) at (2,-3) {};
	 
	 \node (x1) at (-1.5,-3) {$x_1$};
	 \node (x2) at (1.5,-3) {$x_2$};

	\path (root.center) edge node (cutL) [near start]{} (left.center) edge node (cutR) [near start] {}  (right.center);
	\path (left.center) edge (lml.center);
	\path (lml.center) edge (lmm.center);
	\path (lmm.center) edge (lmr.center);
	\path (lmr.center) edge node (cutB) [midway] {} (rml.center);
	\path (rml.center) edge (rmm.center);
	\path (rmm.center) edge (rmr.center);
	\path (rmr.center) edge (right.center);
	\path (cutL.center) edge node (cutC) [midway] {} (cutR.center);
	\path (cutC.center) edge (cutB.center);
\end{tikzpicture}
}
\end{minipage}
 &
 \begin{minipage}{0.3\textwidth}
%\resizebox{1\columnwidth}{3cm}
{
\begin{tikzpicture}[scale=0.5, every node/.style={scale=0.5}]
	\tikzset{decoration={snake,amplitude=1mm,segment length=5mm,
	                       post length=1mm,pre length=1mm}}

	 \node (image) at (-3,-1) {$h_2(r) =$};

	 \node (root) at (0,0) {};
	 \node (left) at (-3,-3) {};
	 \node (right) at (3,-3) {};
	 
	 \node (internalL) at (0,-1.75) {$t'$};
	 
	 \node (lml) at (-2,-3) {};
	 \node (lmm) at (-1.5,-2.5) {};
	 \node (lmr) at (-1,-3) {};
	 
	 \node (rml) at (1,-3) {};
	 \node (rmm) at (1.5,-2.5) {};
	 \node (rmr) at (2,-3) {};
	 
	 \node (x1) at (-1.5,-3) {$x_1$};
	 \node (x2) at (1.5,-3) {$x_2$};

	\path (root.center) edge node (cutL) [near start]{} (left.center) edge node (cutR) [near start] {}  (right.center);
	\path (left.center) edge (lml.center);
	\path (lml.center) edge (lmm.center);
	\path (lmm.center) edge (lmr.center);
	\path (lmr.center) edge node (cutB) [midway] {} (rml.center);
	\path (rml.center) edge (rmm.center);
	\path (rmm.center) edge (rmr.center);
	\path (rmr.center) edge (right.center);
\end{tikzpicture}
}
\end{minipage}
\\
\hline
\parbox{0.1\textwidth}{After} &
\begin{minipage}{0.33 \textwidth}
%\resizebox{1\columnwidth}{3cm}
{
\begin{tikzpicture}[scale=0.5, every node/.style={scale=0.5}]
	\tikzset{decoration={snake,amplitude=1mm,segment length=5mm,
	                       post length=1mm,pre length=1mm}}

	 \node (image) at (-3,-0.4) {$\hat{h}_1(r\up{1}) =$};

	 \node (root) at (0,0) {};
	 \node (left) at (-1.5,-1.5) {};
	 \node (right) at (1.5,-1.5) {};
	 
	 \node (top) at (0,-0.4) {f};
	 
	 \node (lml) at (-1,-1.5) {};
	 \node (lmm) at (-0.5,-1) {};
	 \node (lmr) at (0,-1.5) {};
	 
	 \node (rml) at (0,-1.5) {};
	 \node (rmm) at (0.5,-1) {};
	 \node (rmr) at (1,-1.5) {};
	 
	 \node (x1) at (-0.5,-1) {$x_1$};
	 \node (x2) at (0.5,-1) {$x_2$};
	 
	 \path (top) edge (x1) edge (x2); 

	%\path (root.center) edge node (cutL) [near start]{} (left.center) edge node (cutR) [near start] {}  (right.center);
	%\path (left.center) edge (lml.center);
	%\path (lml.center) edge (lmm.center);
	%\path (lmm.center) edge (lmr.center);
	%\path (rml.center) edge (rmm.center);
	%\path (rmm.center) edge (rmr.center);
	%\path (rmr.center) edge (right.center);
	%\path (lmr.center) edge (rml.center);
	
	\node (lcRoot) at (-0.5,-1.3) {};
	\node (lcLeft)  at (-3.3,-2.8) {};
	\node (lcRight)  at (-0.5,-2.8) {};
	
	\node (lcml) at (-1.8,-2.8) {};
	\node (lcmm) at (-1.3,-2.3) {};
	\node (lcmr) at (-0.8,-2.8) {};
		 
	\node (lcInternal) at (-1,-2) {$t_1$};
	\node (rcInternal) at (1,-2) {$t_2$};
	
	\node (lcLabel) at  (-3,-2) {$\hat{h}_1(r\up{2})=$};
	\node (lcLabel) at  (3,-2) {$=\hat{h}_1(r\up{3})$};
	
	\node (rcRoot) at (0.5,-1.3) {};
	\node (rcLeft)  at (0.5,-2.8) {};
	\node (rcRight)  at (3.3,-2.8) {};
	
	\node (rcml) at (0.8,-2.8) {};
	\node (rcmm) at (1.3,-2.3) {};
	\node (rcmr) at (1.8,-2.8) {};
		 
	\node (x1) at (-1.3,-2.8) {$x_1$};
	\node (x2) at (1.3,-2.8) {$x_1$};
	
	\path (lcRoot.center) edge (lcLeft.center);
	\path (rcRoot.center) edge (rcLeft.center);
	
	\path (lcRoot.center) edge (lcRight.center);
	\path (rcRoot.center) edge (rcRight.center);
	
	\path (lcLeft.center) edge (lcml.center);
	\path (rcLeft.center) edge (rcml.center);
		
	\path (lcml.center) edge (lcmm.center);
	\path (rcml.center) edge (rcmm.center);
	
	\path (lcmm.center) edge (lcmr.center);
	\path (rcmm.center) edge (rcmr.center);
	
	\path (lcmr.center) edge (lcRight.center);
	\path (rcmr.center) edge (rcRight.center);
\end{tikzpicture}
}
\end{minipage}
 &
 \begin{minipage}{0.3 \textwidth}
%\resizebox{1\columnwidth}{3cm}
{
\begin{tikzpicture}[scale=0.5, every node/.style={scale=0.5}]
	\tikzset{decoration={snake,amplitude=1mm,segment length=5mm,
	                       post length=1mm,pre length=1mm}}

	 \node (image) at (-3,-1) {$h_2(r\up{1}) =$};

	 \node (root) at (0,0) {};
	 \node (left) at (-2,-2) {};
	 \node (right) at (2,-2) {};
	 
	 \node (internalL) at (0,-1) {$t'$};
	 
	 \node (lml) at (-1.5,-2) {};
	 \node (lmm) at (-1,-1.5) {};
	 \node (lmr) at (-0.5,-2) {};
	 
	 \node (rml) at (0.5,-2) {};
	 \node (rmm) at (1,-1.5) {};
	 \node (rmr) at (1.5,-2) {};
	 
	 \node (x1) at (-1,-2) {$x_1$};
	 \node (x2) at (1,-2) {$x_2$};

	\path (root.center) edge node (cutL) [near start]{} (left.center) edge node (cutR) [near start] {}  (right.center);
	\path (left.center) edge (lml.center);
	\path (lml.center) edge (lmm.center);
	\path (lmm.center) edge (lmr.center);
	\path (lmr.center) edge node (cutB) [midway] {} (rml.center);
	\path (rml.center) edge (rmm.center);
	\path (rmm.center) edge (rmr.center);
	\path (rmr.center) edge (right.center);
	
	\node (lcLabel) at  (-3,-2.5) {$\hat{h}_2(r\up{2})=$};
	\node (x1) at (-1,-2.5) {$x_1$};
	
	\node (lcLabel) at  (3,-2.5) {$=\hat{h}_2(r\up{3})$};
	\node (x2) at (1,-2.5) {$x_1$};
\end{tikzpicture}
}
\end{minipage}
\\
\hline
\end{tabular}
\end{frame}

\begin{frame}
  \frametitle{The general picture -- 1 Case}
\centering
\begin{tabular}{|c|c|c|}
\hline
& Selected Homomorphism & Unselected Homomorphisms \\
\hline
\parbox{0.1\textwidth}{Before} &
\begin{minipage}{0.33 \textwidth}
%\resizebox{1\columnwidth}{3cm}
{
\begin{tikzpicture}[scale=0.5, every node/.style={scale=0.5}]
	\tikzset{decoration={snake,amplitude=1mm,segment length=5mm,
	                       post length=1mm,pre length=1mm}}

	 \node (image) at (-3,-1) {$h_1(r) =$};

	 \node (root) at (0,0) {};
	 \node (left) at (-3,-3) {};
	 \node (right) at (3,-3) {};
	 
	 \node (top) at (0,-0.4) {f};
	 \node (internalL) at (-0.75,-1.75) {$t_1$};
	 \node (internalR) at (1.15,-1.75) {$t_2$};	  
	 
	 \node (lml) at (-2.6,-3) {};
	 \node (lmm) at (-2.1,-2.5) {};
	 \node (lmr) at (-1.6,-3) {};
	 
	 \node (rml) at (-1.1,-3) {};
	 \node (rmm) at (-0.6,-2.5) {};
	 \node (rmr) at (-0.1,-3) {};
	 
	 \node (x1) at (-2.1,-3) {$x_1$};
	 \node (x2) at (-0.6,-3) {$x_2$};
	 
	 \node (it) at (0,-3) {};

	\path (root.center) edge node (cutL) [near start]{} (left.center) edge node (cutR) [near start] {}  (right.center);
	\path (left.center) edge (lml.center);
	\path (lml.center) edge (lmm.center);
	\path (lmm.center) edge (lmr.center);
	\path (lmr.center) edge (rml.center);
	\path (rml.center) edge (rmm.center);
	\path (rmm.center) edge (rmr.center);
	\path (rmr.center) edge node (cutB) [near start] {} (right.center);
	\path (cutL.center) edge node (cutC) [midway] {} (cutR.center);
	\path (cutC.center) edge (it.center);
\end{tikzpicture}
}
\end{minipage}
 &
 \begin{minipage}{0.2 \textwidth}
%\resizebox{1\columnwidth}{3cm}
{
\begin{tikzpicture}[scale=0.3, every node/.style={scale=0.5}]
	\tikzset{decoration={snake,amplitude=1mm,segment length=5mm,
	                       post length=1mm,pre length=1mm}}

	 \node (image) at (-3,-1) {$h_2(r\up{1}) =$};

	 \node (root) at (0,0) {};
	 \node (left) at (-3,-3) {};
	 \node (right) at (3,-3) {};
	 
	 \node (internalL) at (0,-1.75) {$t'$};
	 
	 \node (lml) at (-2,-3) {};
	 \node (lmm) at (-1.5,-2.5) {};
	 \node (lmr) at (-1,-3) {};
	 
	 \node (rml) at (1,-3) {};
	 \node (rmm) at (1.5,-2.5) {};
	 \node (rmr) at (2,-3) {};
	 
	 \node (x1) at (-1.5,-3) {$x_1$};
	 \node (x2) at (1.5,-3) {$x_2$};

	\path (root.center) edge node (cutL) [near start]{} (left.center) edge node (cutR) [near start] {}  (right.center);
	\path (left.center) edge (lml.center);
	\path (lml.center) edge (lmm.center);
	\path (lmm.center) edge (lmr.center);
	\path (lmr.center) edge node (cutB) [midway] {} (rml.center);
	\path (rml.center) edge (rmm.center);
	\path (rmm.center) edge (rmr.center);
	\path (rmr.center) edge (right.center);
\end{tikzpicture}
}
\end{minipage}
\\
\hline
\parbox{0.1\textwidth}{After} &
\begin{minipage}{0.33 \textwidth}
%\resizebox{1\columnwidth}{3cm}
{
\begin{tikzpicture}[scale=0.5, every node/.style={scale=0.5}]
	\tikzset{decoration={snake,amplitude=1mm,segment length=5mm,
	                       post length=1mm,pre length=1mm}}

	 \node (image) at (-3,-0.4) {$\hat{h}_1(r\up{1}) =$};

	 \node (root) at (0,0) {};
	 \node (left) at (-1.5,-1.5) {};
	 \node (right) at (1.5,-1.5) {};
	 
	 \node (top) at (0,-0.4) {f};
	 
	 \node (lml) at (-1,-1.5) {};
	 \node (lmm) at (-0.5,-1) {};
	 \node (lmr) at (0,-1.5) {};
	 
	 \node (rml) at (0,-1.5) {};
	 \node (rmm) at (0.5,-1) {};
	 \node (rmr) at (1,-1.5) {};
	 
	 \node (x1) at (-0.5,-1) {$x_1$};
	 \node (x2) at (0.5,-1) {$x_2$};
	 
	 \path (top) edge (x1) edge (x2); 

	%\path (root.center) edge node (cutL) [near start]{} (left.center) edge node (cutR) [near start] {}  (right.center);
	%\path (left.center) edge (lml.center);
	%\path (lml.center) edge (lmm.center);
	%\path (lmm.center) edge (lmr.center);
	%\path (rml.center) edge (rmm.center);
	%\path (rmm.center) edge (rmr.center);
	%\path (rmr.center) edge (right.center);
	%\path (lmr.center) edge (rml.center);
	
	\node (lcRoot) at (-0.5,-1.8) {};
	\node (lcLeft)  at (-3.3,-3.3) {};
	\node (lcRight)  at (-0.5,-3.3) {};
	
	\node (lcml) at (-2.7,-3.3) {};
	\node (lcmm) at (-2.2,-2.8) {};
	\node (lcmr) at (-1.7,-3.3) {};
		 
	\node (lcInternal) at (-1,-2.5) {$t_1$};
	\node (rcInternal) at (1,-2.5) {$t_2$};
	
	\node (lcLabel) at  (-3,-2.5) {$\hat{h}_1(r\up{2})=$};
	\node (lcLabel) at  (3,-2.5) {$=\hat{h}_1(r\up{3})$};
	
	\node (rcRoot) at (0.5,-1.8) {};
	\node (rcLeft)  at (0.5,-3.3) {};
	\node (rcRight)  at (3.3,-3.3) {};
	
	\node (rcml) at (-1.6,-3.3) {};
	\node (rcmm) at (-1.1,-2.8) {};
	\node (rcmr) at (-0.6,-3.3) {};
		 
	\node (x1) at (-2.2,-3.3) {$x_1$};
	\node (x2) at (-1.1,-3.3) {$x_2$};
	
	\path (lcRoot.center) edge (lcLeft.center);
	\path (rcRoot.center) edge (rcLeft.center);
	
	\path (lcRoot.center) edge (lcRight.center);
	\path (rcRoot.center) edge (rcRight.center);
	
	\path (lcLeft.center) edge (lcml.center);
	\path (rcLeft.center) edge (rcRight.center);
		
	\path (lcml.center) edge (lcmm.center);
	\path (rcml.center) edge (rcmm.center);
	
	\path (lcmm.center) edge (lcmr.center);
	\path (rcmm.center) edge (rcmr.center);
	
	\path (lcmr.center) edge (rcml.center);
	\path (rcmr.center) edge (lcRight.center);
\end{tikzpicture}
}
\end{minipage}
 &
 \begin{minipage}{0.23 \textwidth}
%\resizebox{1\columnwidth}{3cm}
{
\begin{tikzpicture}[scale=0.5, every node/.style={scale=0.5}]
	\tikzset{decoration={snake,amplitude=1mm,segment length=5mm,
	                       post length=1mm,pre length=1mm}}

	 \node (image) at (-2,0.5) {$h_2(r\up{1}) =$};
	 \node (image) at (-2,-1) {$h_2(r\up{2}) =$};

	 \node (point) at (0,0.5) {$x_1$};

	 \node (root) at (0,0) {};
	 \node (left) at (-2,-2) {};
	 \node (right) at (2,-2) {};
	 
	 \node (internalL) at (0,-1) {$t'$};
	 
	 \node (lml) at (-1.5,-2) {};
	 \node (lmm) at (-1,-1.5) {};
	 \node (lmr) at (-0.5,-2) {};
	 
	 \node (rml) at (0.5,-2) {};
	 \node (rmm) at (1,-1.5) {};
	 \node (rmr) at (1.5,-2) {};
	 
	 \node (x1) at (-1,-2) {$x_1$};
	 \node (x2) at (1,-2) {$x_2$};

	\path (root.center) edge node (cutL) [near start]{} (left.center) edge node (cutR) [near start] {}  (right.center);
	\path (left.center) edge (lml.center);
	\path (lml.center) edge (lmm.center);
	\path (lmm.center) edge (lmr.center);
	\path (lmr.center) edge node (cutB) [midway] {} (rml.center);
	\path (rml.center) edge (rmm.center);
	\path (rmm.center) edge (rmr.center);
	\path (rmr.center) edge (right.center);
	
	\node (lcLabel) at  (3,-1) {$=\hat{h}_1(r\up{3})$};
	\node (x2) at (2,-1) {$\top$};
\end{tikzpicture}
}
\end{minipage}
\\
\hline
\end{tabular}
\end{frame}


\section{Consequences}

\begin{frame}
  \frametitle{Main consequences:}
\begin{itemize}
\item Any bimorphism/IRTG with linear homorphism and binary signatures can be brought to TNF.
\end{itemize}

\begin{block}{Thm}
  The translation problem for an $LB$ bimorphism is at most linear in the size of the input FTA $F$.
\end{block}

\begin{block}{Thm}
  Linear homomorphisms over binary signatures do not induce an overhead in parsing/translation time: if an IRTG is based on an $LB$ bimorphism, then an input object $o$ can be translated in time $|D_o|$, if $D_o$ is a decomposition automaton for $o$.
\end{block}
\end{frame}

\begin{frame}
  \frametitle{Further remarks:}
Moreover:
\begin{itemize}
\item TNF reduction is adaptable to the weighted case.
\item Bringing an IRTG into TNF takes time (hence space) proportional to the size of the original IRTG.
\item In C.L., one almost always relies on 'known' algebras for which the cost of decomposition is known.
\item Hence, one can immediately conclude to a parsing/translation complexity upper bound, for any new formalism encoded as an IRTG using such an algebra and $LB$-bimorphisms.
\item TNF reduction can be combined with existing binarization procedures. 
\end{itemize}
\end{frame}



\section{Conclusion}

\begin{frame}
  \frametitle{Conclusion and future directions.}
  \begin{itemize}
  \item A truncated normal form for $LB$ bimorphisms and IRTG based on these.
  \item Yields a complexity result on translating with $LB$-bimorphisms.
  \item As a corrolary, IRTGs based on $LB$ bimorphisms have complexity upper bound depending on the input algebra only.
  \end{itemize}

  Research perspectives:
  \begin{itemize}
  \item TNF reduces the search space for grammar induction (learning a weighted IRTG).
  \item Could yield more efficient parsing strategies (\emph{e.g.}, greedy statistical parsers).
  \end{itemize}
\end{frame}
 

\end{document}
