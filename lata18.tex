\documentclass{beamer}

%\usepackage{listings}
%\usepackage[francais]{babel}
\usepackage[T1]{fontenc}
\usepackage[utf8]{inputenc}
%\usepackage{MyriadPro}
\usepackage{cabin}
\usepackage{graphicx}
\usepackage{array}
\usepackage{tikz}
\usetikzlibrary{positioning, backgrounds, shapes, chains, decorations.pathmorphing}
\usepackage{amsmath,amsthm,amssymb}  
\usepackage{stmaryrd}
%\usepackage{mdsymbol}
\usepackage{MnSymbol}
\usepackage{xcolor}
\usepackage{verbatim}
\usepackage{array}
%\usepackage{csquotes}


\useoutertheme{infolines}

\newcommand{\hidden}[1]{}

%colors
\definecolor{darkgreen}{rgb}{0,0.5,0}
\usebeamercolor{block title}
\definecolor{beamerblue}{named}{fg}
\usebeamercolor{alert block title}
\definecolor{beamealert}{named}{fg}

\renewcommand{\colon}{\!:\!}


\newcommand\paraitem{%
 \quad
 \makebox[\labelwidth][r]{%
 \makelabel{%
 \usebeamertemplate{itemize \beameritemnestingprefix item}}}\hskip\labelsep}

\newcommand{\mmid}{\mathbin{{\mid}{\mid}}}

\begin{document}

\title{Efficient Translation with Linear Bimporphisms} 
\author[Antoine Venant]{Christoph Teichmann \and Antoine Venant \and Alexander Koller}
\institute{Saarland University}
\date{\today}
\maketitle


\section{Introduction: the translation problem.}

\begin{frame}
  \frametitle{What this talk is about -- Bimorphisms}
  \begin{itemize}
  \item \textbf{Bimorphisms} offer a convenient and succint way to define relation/transformations between trees.
  \item Model syntax-based relations between objects.
  \item Idea: two trees are related if both are images of a third \textbf{derivation} tree, under two distinct homomorphisms.
  \item admissible derivation trees are compactly specified as a regular tree language. 
  \end{itemize}
\end{frame}




%\subsection{Translating with bimorphisms}



%\subsection{Invers homomorphism problem}

%\begin{frame}
%  \frametitle{?}
%\end{frame}

%\section{Solution sketch & further motivations}

%\section{Truncation Step}





%\begin{frame}

\end{document}
